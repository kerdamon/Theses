\chapter{Conclusions}

The results of the simulations performed largely meet the expectations and agree with the logic and analytical models. An example is the fact that in a favorable environment, agents experienced significant growth, while the population vary under more extreme conditions. Interestingly, it was the harsh conditions that drove the genetic development of the agents, because it was then that weaker individuals were rejected by natural selection. The conclusion is that the conditions for reproduction must be ensured, as well as the danger of discarding the weakest individuals, for genetic development to proceed.

On the other hand, the simulation has some puzzling shortcomings, for example the final values of the fox traits in \nameref{Simulation4}. In this case, it is surprising that a higher speed value, for example, would not be beneficial for the foxes. This may, of course, occurred due to the quality of the trained models. For example, in most of the simulations, the agents had a problem with collisions with water, and avoiding them was problematic for them. The second reason for weaknesses in the simulation may be that while the number of agents in the simulation was sufficient to observe interesting behavior, it was not large due to technical reasons. With so many agents, the simulation was already heavily loaded and it was demanding on hardware.

Nevertheless, the simulations carried out can definitely be considered successful. Its very unpredictability caused by randomness makes it possible to observe interesting behaviors that cannot be obtained in analytical models. Additionally, the presented model of the environment can be freely developed, increasing its possibilities, which is described in the \autoref{furtherDevelopmentPossibilitiesChapter}. 