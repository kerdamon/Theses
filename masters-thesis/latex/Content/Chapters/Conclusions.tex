\chapter{Conclusions} \label{chr:conclusions}

The objective set out in the introduction has been met. The tests conducted demonstrate that the developed modules operate as intended and can be used to perform the Winston-Lutz test and leaf alignment analysis.

The fact that the tools have been developed as part of a web application makes them highly accessible for use on any device with access to a modern browser. Moreover, the application's open-source nature opens up a wide range of possible applications in situations where it was previously impractical, frequently because of resource constraints.

However, it should be mentioned that the application is currently in the prototype stage and has not been certified. As such, it should not be applied to the calibration of medical equipment meant to be utilized during procedures on human beings. At this stage of development, the application is only a proof of concept. Nevertheless, it does demonstrate the potential of such solutions, which, with sufficient effort, could greatly facilitate the operation of numerous facilities utilising linear accelerators, given that such devices will require calibration regularly.

The analysis of the results of the W-L tests reveals that in instances where the test is expected to yield accurate results, it does so. Furthermore, in cases that should identify the machine as uncalibrated, results also match. This proves that the module has been correctly embedded in the application. 

Observing the analysis of the results for LA in an analogous way shows the correct implementation of the module, giving the expected results. Further interpretation, however, leads to the conclusion that the more complex the shape, the greater the problem the edge detection algorithm will have, which can lead to erroneous indications. 

The outcomes of the work has been reviewed by Damian Kubat, the head of the Department of Medical Physics at the National Institute of Oncology in Kraków. He provided a positive feedback on its implementation and indicates that it is a valuable tool, especially if developed further.
